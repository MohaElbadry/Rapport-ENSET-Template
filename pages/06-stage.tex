\chapter{Description du stage}
% \renewcommand{\thesection}{\arabic{section}} % Numeric numbering for sections
\renewcommand{\thesubsection}{\arabic{subsection}} % Numeric numbering for subsections
% \renewcommand{\thesubsubsection}{\arabic{subsubsection}} % Numeric numbering for subsections
\markboth{ Réaslisation de projet}{Réaslisation de projet}
\label{part:Réaslisation-de-projet}
La présente section se concentre sur la phase de mise en œuvre de la solution, en mettant en avant les diverses interfaces qui composent le projet, ainsi que les fonctionnalités spécifiques de chaque interface.\\

\section{Description du projet}

Le projet auquel j'ai contribué avait pour but de développer une application web et mobile visant à optimiser la gestion des processus internes de l'entreprise. Cette application devait offrir une interface utilisateur fluide et intuitive, tout en intégrant des fonctionnalités avancées pour la gestion des données en temps réel. Le projet s'inscrivait dans une stratégie globale de transformation numérique de l'entreprise, afin de moderniser ses outils et de répondre aux exigences croissantes de ses clients. Le développement de cette application m'a permis de me familiariser avec de nouvelles technologies et d'acquérir une expertise en architecture logicielle.

\section{Cahier des charges}

Le cahier des charges du projet définissait les attentes précises de l'entreprise ainsi que les fonctionnalités clés à implémenter. Il était essentiel de respecter les délais impartis et de garantir la qualité des livrables à chaque phase du projet. En suivant ce cahier des charges, j'ai pu mieux comprendre les besoins du client et les contraintes spécifiques à ce type de développement.

\subsection{Objectifs du projet}

L'objectif principal du projet était de fournir à l'entreprise un outil permettant de centraliser la gestion des processus, d'améliorer la communication entre les différents départements et d'automatiser certaines tâches répétitives. Il s'agissait également de renforcer la sécurité des données et d'offrir une meilleure traçabilité des actions effectuées sur la plateforme. De plus, l'application devait être évolutive pour pouvoir intégrer de nouvelles fonctionnalités à l'avenir.

\subsection{Besoins fonctionnels}

Les besoins fonctionnels du projet incluaient l'authentification sécurisée des utilisateurs, la gestion des rôles et des permissions, la génération de rapports en temps réel, ainsi qu'un tableau de bord interactif pour le suivi des performances. L'application devait également intégrer une fonctionnalité de notification, permettant aux utilisateurs de recevoir des alertes sur les événements importants liés à leurs activités.

\section{Environnement de développement}

Le développement du projet s'est effectué dans un environnement technique moderne, composé des dernières technologies en matière de développement web et mobile. L'outil principal utilisé pour le développement de l'application était \texttt{React.js} pour la partie frontend et \texttt{Node.js} pour la partie backend. Les bases de données étaient gérées avec \texttt{MongoDB}, permettant un stockage efficace des données en temps réel. Les systèmes de contrôle de version, tels que \texttt{Git}, étaient utilisés pour garantir une gestion cohérente des versions du code et faciliter le travail en équipe. Des outils de collaboration comme \texttt{Jira} et \texttt{Slack} ont permis de suivre l'avancement du projet et de faciliter la communication entre les membres de l'équipe.


\begin{figure}[h]
    \centering
    \includegraphics[width=150px]{LOGOS/Slack-Logo.png}
    \caption{Logo Slack}
    \label{fig:slack}
\end{figure}

\textbf{Slack:} Slack est une plateforme de communication collaborative et un logiciel de gestion de projets créé par Stewart Butterfield. Il inclut des fonctionnalités de chat entre utilisateurs ou en groupe, de partage de documents et de recherche de contenu.

\begin{figure}[h]
    \centering
    \includegraphics[width=80px]{LOGOS/vs.png}
    \caption{Logo Visual Studio Code}
    \label{fig:vscode}
\end{figure}

\textbf{Visual Studio Code:} Visual Studio Code est un éditeur de code source développé par Microsoft pour Windows, Linux et macOS. Il comprend la prise en charge du débogage, du contrôle Git intégré et de GitHub, de la coloration syntaxique, de la complétion de code intelligente, et bien plus encore.
\break
\section{Conclusion}
En résumé, cette section a permis de mettre en lumière les étapes de conception et de modélisation du système, en utilisant la méthode MERISE. Ces modèles conceptuels et physiques sont essentiels pour assurer la cohérence, l'efficacité et la fiabilité de la gestion des données dans l'application. Ils constituent une base solide pour la poursuite du développement et de l'implémentation de notre solution.