% \chapter{Présentation Contexte du projet }
\renewcommand{\chaptername}{Partie}

\label{part:introduction-generale}
\chapter{Présentation de l'entreprise}
\section{Introduction }
Dans cette section, nous présentons le projet que nous avons réalisé. Ce projet répondait aux exigences spécifiées dans le cahier des charges, que nous allons détailler dans la première partie. Nous expliquerons ensuite les outils techniques  que nous avons utilisés pour mener à bien ce projet. Enfin, nous présenterons l'ensemble des \\

\section{Présentation de l’organisme d’accueil}
L’organisme d’accueil joue un rôle crucial dans le cadre de mon stage d’initiation. Il s'agit d'une entreprise dynamique et innovante, spécialisée dans le développement de solutions technologiques de pointe. Fondée sur des valeurs telles que l'excellence, la rigueur et l'innovation, l'entreprise a su se positionner comme un acteur clé dans son secteur d'activité. Grâce à une équipe pluridisciplinaire composée d'experts hautement qualifiés, l’organisme d’accueil se distingue par sa capacité à répondre aux besoins variés de ses clients en proposant des services personnalisés et à forte valeur ajoutée.

Son champ d’expertise s’étend sur plusieurs domaines, notamment le développement de logiciels sur mesure, l'intégration de systèmes d'information, le cloud computing, ainsi que le Big Data. L'entreprise a su anticiper les évolutions technologiques récentes pour offrir des solutions adaptées aux défis actuels, en particulier dans les secteurs liés à la transformation numérique des entreprises. \\

\begin{figure}[h]
\centering
\includegraphics[width=0.5\textwidth]{LOGOS/enset_mohammedia_0.png}
\caption{Logo Enset}
\label{fig:image-label}
\end{figure}

\section{ Présentation des services }
\medskip

\textbf{Le développement web :} 
Concerne le développement des sites Web destinés à être hébergé sur internet, ce service tient à répondre aux attentes du client, en passant de la conception, la créationdu contenu, à l’hébergement, et le suivi.\\
\medskip

\textbf{Le développement Desktop :}
 Concerne les applications installées sur pc, son but est d’aiderses clients à digitaliser leur service pour une gestion plus efficace.\\
 \medskip

\textbf{Le développement mobile }
: Concerne le développement des applications mobiles qui
sont mises au niveau des plateformes Android / ios, et ont pour but à garder son consommateur plus procheà la supervision de ses activités à tout temps.\\
\medskip

\textbf{E-commerce :}
 ou commerce en ligne,  digital assure à ses clients une bonne insertion dans le monde du commerce digital, en suivant tout un processus de développement commençant par la stratégie marketing, et en créant une boutique en ligne hébergée sur le web commel’exemple de YaneCode robotique\\
 \medskip

\textbf{Iot :}
Internet Of Things, concerne tout ce qui a relation avec la robotique, ils
proposent des workshops sur place avec  Academy, et aussi des solutions
robotiques Smart pour permettre à ses clients à bénéficier du monde de l’intelligence artificielle.\\
\medskip

\section{ Conclusion}
L'étude préliminaire a été cruciale pour identifier les problématiques du sujet, définir la méthodologie de travail et choisir les données pertinentes pour la modélisation de la base de données, ainsi que pour l'analyse et la conception de l'application.

